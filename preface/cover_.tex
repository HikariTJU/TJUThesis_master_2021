% % !Mode:: "TeX:UTF-8"

% %%  可通过增加或减少 setup/format.tex中的
% %%  第274行 \setlength{\@title@width}{8cm}中 8cm 这个参数来 控制封面中下划线的长度。

% \cheading{天津大学~2016~届本科生毕业论文}      % 设置正文的页眉,需要填上对应的毕业年份
% \ctitle{基于顾客有限理性预期的定价与供应链结构}    % 封面用论文标题,自己可手动断行
% \caffil{管理与经济学部} % 学院名称
% \csubject{工业工程}   % 专业名称
% \cgrade{2012~级}            % 年级
% \cauthor{秦昱博}            % 学生姓名
% \cnumber{3012209017}        % 学生学号
% \csupervisor{杨道箭}        % 导师姓名
% \crank{副教授}              % 导师职称

% \cdate{\the\year~年~\the\month~月~\the\day~日}

% \cabstract{
% 中文摘要一般在~400~字以内,简要介绍毕业论文的研究目的、方法、结果和结论,语言力求精炼。中英文摘要均要有关键词,一般为~3~—~7~个。字体为小四号宋体,各关键词之间要有分号。英文摘要应与中文摘要相对应,字体为小四号~Times New Roman,详见模板。
% }

% \ckeywords{关键词~1;关键词~2;关键词~3;……;关键词~7(关键词总共~3~—~7~个,最后一个关键词后面没有标点符号)}

% \eabstract{
% The upper bound of the number of Chinese characters is 400. The abstract aims at introducing the research purpose, research methods, research results, and research conclusion of graduation thesis, with refining words. Generally speaking, both the Chinese and English abstracts require the keywords, the number of which varies from 3 to 7, with a semicolon between adjacent words. The font of the English Abstract is Times New Roman, with the size of 12pt(small four).
% }

% \ekeywords{keyword 1, keyword 2, keyword 3, ……, keyword 7 (no punctuation at the end)}

% \makecover

% \clearpage


% !Mode:: "TeX:UTF-8"


\ctitle{基于贯口的传统相声研究}  %封面用论文标题,自己可手动断行
\etitle{Research on traditional cross talk based on Guankou}
\caffil{天津大学电气自动化与信息工程学院} %学院名称
\cfirstsubjecttitle{\textbf{一级学科}}
\cfirstsubject{\textbf{\underline{\makebox[14em][c]{信息与通信工程}}}}   %专业
\csubjecttitle{\textbf{研究方向}}
\csubject{\textbf{\underline{\makebox[14em][c]{信号与信息处理}}}}   %专业
\cauthortitle{\textbf{作者姓名}}     % 学位
\cauthor{\textbf{\underline{\makebox[14em][c]{WH}}}}   %学生姓名
\csupervisortitle{\textbf{指导教师}}
\csupervisor{\textbf{\underline{\makebox[14em][c]{**~~副教授}}}} %导师姓名

\teachertable{
\begin{table}[h]
\centering
\song\xiaosi{
\begin{tabularx}{\textwidth}{|*{4}{>{\centering\arraybackslash}X|}}
\hline
\textbf{答辩日期}                & \multicolumn{3}{c|}{20   年   月   日}       \\ \hline
\textbf{答辩委员会}               & \textbf{姓名} & \textbf{职称} & \textbf{工作单位} \\ \hline
\textbf{主席}                  &             &             &               \\ \hline
\multirow{2}{*}{\textbf{委员}} &             &             &               \\ \cline{2-4} 
                             &             &             &               \\ \hline
\end{tabularx}}
\end{table}}


\declaretitle{独创性声明}
\declarecontent{
本人声明所呈交的学位论文是本人在导师指导下进行的研究工作和取得的研究,成, 果,除了文中特别加以标注和致谢之处外,论文中不包含其他人已经发表或撰写过的研究成果,也不包含为获得 {\underline{\kaiGB{\sihao{\textbf{~~天津大学~~}}}}} 或其他教育机构的学位或证书而使用过的材料。与我一同工作的同志对本研究所做的任何贡献均已在论文中作了明确的说明并表示了谢意。
}
\authorizationtitle{学位论文版权使用授权书}
\authorizationcontent{
本学位论文作者完全了解{\underline{\kaiGB{\sihao{\textbf{~~天津大学~~}}}}}有关保留、使用学位论文的规定。特授权{\underline{\kaiGB{\sihao{\textbf{~~天津大学~~}}}}} 可以将学位论文的全部或部分内容编入有关数据库进行检索,并采用影印、缩印或扫描等复制手段保存、汇编以供查阅和借阅。同意学校向国家有关部门或机构送交论文的复印件和磁盘。
}
\authorizationadd{(保密的学位论文在解密后适用本授权说明)}
\authorsigncap{学位论文作者签名:}
\supervisorsigncap{导师签名:}
\signdatecap{签字日期:}


\cdate{\CJKdigits{\the\year} 年\CJKnumber{\the\month} 月 \CJKnumber{\the\day} 日}
% 如需改成二零一二年四月二十五日的格式,可以直接输入,即如下所示
\cdate{二零二一年九月}
% \cdate{\the\year 年\the\month 月 \the\day 日} % 此日期显示格式为阿拉伯数字 如2012年4月25日
\cabstract{

八扇屏是传统相声的一部分。八扇屏是传统相声的一部分。八扇屏是传统相声的一部分。八扇屏是传统相声的一部分。八扇屏是传统相声的一部分。

}

\ckeywords{关键词,关键词,关键词,关键词,关键词}

\eabstract{

The eight screens are part of the traditional comic dialogue. The eight screens are part of the traditional comic dialogue. The eight screens are part of the traditional comic dialogue. The eight screens are part of the traditional comic dialogue. 

}

\ekeywords{Keyword, Keyword, Keyword, Keyword, Keyword}

\makecover
\clearpage